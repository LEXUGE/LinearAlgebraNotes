% Created 2020-05-17 Sun 21:29
% Intended LaTeX compiler: pdflatex
\documentclass[11pt]{article}
\usepackage[utf8]{inputenc}
\usepackage[T1]{fontenc}
\usepackage{graphicx}
\usepackage{grffile}
\usepackage{longtable}
\usepackage{wrapfig}
\usepackage{rotating}
\usepackage[normalem]{ulem}
\usepackage{amsmath}
\usepackage{textcomp}
\usepackage{amssymb}
\usepackage{capt-of}
\usepackage{hyperref}
\hypersetup{colorlinks=true,linkcolor=blue}
\usepackage[margin=0.5in]{geometry}
\author{Harry Ying}
\date{}
\title{Linear Algebra Note}
\hypersetup{
 pdfauthor={Harry Ying},
 pdftitle={Linear Algebra Note},
 pdfkeywords={},
 pdfsubject={},
 pdfcreator={Emacs 26.3 (Org mode 9.1.9)},
 pdflang={English}}
\begin{document}

\maketitle
\tableofcontents \clearpage
\section{Chapter 1}
\label{sec:org8d1da7e}
\subsection{\(\S\) 1}
\label{sec:orgdb2d4a2}
\subsubsection{Notes}
\label{sec:org54f9af6}
A generic vector space \(V\) is not a field because there is no definition of \(v^{-1}\) for some \(v\in V\), fulfilling not the definition of a field.\\
\begin{enumerate}
\item \textbf{Pg. 4 Proof of \((-1)v=v\)}
\label{sec:orgaca242a}
$$\begin{aligned}
(-1)v+v&=(-1)v+1\cdot v\\
&=(-1+1)v\\
&=v+(-v)
\end{aligned}$$
Thus, \((-1)v=-v\).
\item \textbf{Pg. 6 Proof of SP 3}
\label{sec:org6bbf976}
$$\begin{aligned}
(xA)\cdot B&=\sum\limits_{i=1}^{n}(xa_i) b_{i}\\
&=\sum\limits_{i=1}^{n}x(a_i b_{i})\\
&=x\sum\limits_{i=1}^{n} a_i b_{i}\\
&=x(A\cdot B)\\
A\cdot (xB)&=\sum\limits_{i=1}^{n}a_i (xb_{i})\\
&=\sum\limits_{i=1}^{n}x(a_i b_{i})\\
&=x\sum\limits_{i=1}^{n} a_i b_{i}\\
&=x(A\cdot B)
\end{aligned}$$
\item \textbf{\textbf{Pg. 7}}
\label{sec:org900c222}

Upper one:
$$\begin{aligned}
(A+B)^2&=(A+B)\cdot (A+B)\\
&=(A+B)\cdot A+(A+B)\cdot B && \text{Use SP 2}\\
&=A^2+B\cdot A+A\cdot B+B^2 && \text{Use SP 1}\\
\end{aligned}$$
Bottom one:
Since \(K\) is a field, all \textbf{VS} s regarding summation or product of functions are actually closed on \(K\). By applying field axioms, \(V\) is then a vector space over \(K\).
\item \textbf{\textbf{Pg. 9}}
\label{sec:orgdb92e0c}

\label{orgb5f42fa}
Let \(a_1=(u_1+w_1),a_2=(u_2+w_2)\). Both of them \(\in (U+W)\).\\
Since \(U,W\) are subspaces of \(V\), \(U,W\in V\). Thus, \(a_1,a_2 \in V\) as \(u_1,w_1,u_2,w_2\in V\), moreover, \((U+W)\subset V\).\\
\(a_1+a_2=(u_1+u_2)+(w_1+w_2)\in (U+W)\) \\
\(ca_1=c(u_1+w_1)=(cu_1)+(cw_1)\in (U+W)\) \\
Since \(O\in U\) and \(O\in W\), \(O=O+O\in (U+W)\). Thus, \((U+W)\) is a subspace of \(V\).
\end{enumerate}
\subsubsection{Exercises}
\label{sec:org4bb2bda}
\begin{enumerate}
\item \textbf{Exercise 1}
\label{sec:org147b898}
Let \(v\in{} V\), \(c[v+(-v)]=cv+c(-v)=cv+(-c)v=v\cdot{}0=v\cdot{}(1-1)=v+(-v)=O\)
\item \textbf{Exercise 2}
\label{sec:orgc5771d1}
Since \(c\not = 0\)
$$\begin{aligned}
O&=cv+[-(cv)]\\
cv&=cv+[-(cv)]\\
O&=-(cv)\\
\frac{-1}{c}\cdot O &= (-c)v\cdot \frac{-1}{c}\\
\frac{-1}{c}\cdot (v-v) &= v\\
\frac{-1}{c}\cdot v+ \frac{1}{c}\cdot v &= v\\
v\cdot (1-1)&=v\\
v-v&=v\\
O&=v
\end{aligned}$$
\item \textbf{Exercise 3}
\label{sec:org8e6ea32}

\(\forall g\in V, (g+f)(x) = g(x)+f(x) = f(x)+g(x) = (f+g)(x) \Rightarrow g+f = f+g\).\\
If \(O+u = u\), \((O+u)(x) = O(x)+u(x)= u(x)\). Therefore, \(O(x)=0\).
\item \textbf{Exercise 4}
\label{sec:orgfcbed1e}
$$\begin{aligned}
v+w&=O\\
v+w&=v+(-v)\\
w&=-v
\end{aligned}$$
\item \textbf{Exercise 5}
\label{sec:org094d985}
$$\begin{aligned}
v+w&=v\\
v+(-v)+w&=v+(-v)\\
O+w&=O
\end{aligned}$$
Since \(\forall u, O+u=u\), we have \(w=O\).
\item \textbf{Exercise 6}
\label{sec:orgcd29776}

Let \(W=\{B| B\cdot A_{1}=O\ \text{and}\ B\cdot A_2=O\}\). Specifically, it is clear that \(O\in W\) as \(O\cdot A = \sum\limits_{i=1}^{n} b_i a_i=\sum\limits_{i=1}^{n} 0\times a_i=0\).\\
Let \(v_1,v_2 \in W\) such that \(v_1\cdot A_1=0\), \(v_1\cdot A_2=0\), \(v_2\cdot A_1=0\), \(v_2\cdot A_2=0\). Thus,
$$\begin{aligned}
(v_1+v_2)\cdot A_1&=v_1\cdot A_1+v_2\cdot A_1\\
&=O+O\\
&=O\\
[c(v_1+v_2)]\cdot A_1&=(cv_1+cv_2)\cdot A_1\\
&=(cv_1)\cdot A_1+(cv_2)\cdot A_1\\
&=c(v_1\cdot A_1+v_2\cdot A_1)\\
&=cO\\
&=O
\end{aligned}$$.
It is easy to show for \(A_2\) then. Therefore, \((v_1+v_2)\in W\).
\item \textbf{Exercise 7}
\label{sec:orgdb9ee0e}
Same to apply as Exercise 6.
\item \textbf{Exercise 8}
\label{sec:orgb70bb37}

Name the set as \(W\).
\begin{enumerate}
\item Proof
\label{sec:org3601c12}

\(v_1+v_2=(x_1+x_2,y_1+y_2), x_1+x_2=y_1+y_2 \Rightarrow (v_1+v_2)\in W\) \\
\(cv=(cx,cy), cx=cy \Rightarrow cv\in W\) \\
\(O=(0,0)\in W\)
\item Proof
\label{sec:org06bdcc1}
See Part (a).
\item Proof
\label{sec:org51a6d2a}
Same technique as in Part (a).
\end{enumerate}
\item \textbf{Exercise 9}
\label{sec:orgc4a8483}
See Exercise 8.
\item \textbf{Exercise 10}
\label{sec:orgdc983d4}

For \(U\cap W\), let \(v_1,v_2\in U\cap W\). Since \(v_1, v_2\in U\) and \(U\) is a subspace, \(v_1+v_2\in U\). In same way, we can see that \(v_{1}+v_2\in W\). Thus, \(v_1+v_2\in U\cap W\).\\
Since \(v_1\in U\), \(cv_1\in U\). Also, it shows \(cv_1\in W\) in the same way. Thus, \(cv_{1}\in U\cap W\).
Because \(U, W\) are subspaces, \(O\in U\) and \(O\in W\). Thus, \(O\in U\cap W\). Therefore, \(U\cap W\) is a subspace.\\
Refer to the \hyperref[orgb5f42fa]{note part} for proof for \(U+W\).
\item \textbf{Exercise 11}
\label{sec:org2c08fa7}
Since \(L\) is a field, \textbf{VS1, VS3, VS4, VS8} are established under field axioms, and multiplication and addition are closed in \(L\). For \textbf{VS5, VS6, VS7}, they are all valid as \(K\subset L\). \(O\) is simply \(0\), and \(1\cdot u=u\) is  established in \(L\).
\item \textbf{Exercise 12}
\label{sec:org4597395}

For \(x,y\in K\), we have\\
\(x+y=a_1+b_1\sqrt{2}+a_2+b_2\sqrt{2}=(a_1+a_2)+(b_1+b_2)\sqrt{2}\). Since \(a_1,b_1,a_2,b_2\in \mathbb{Q}\), \((a_1+a_2),(b_1+b_2)\in\mathbb{Q}\). Thus, \(x+y\in K\).\\
\(xy=(a_1 a_2+ 2b_1 b_2)+(a_2 b_1 + a_1 b_2)\times \sqrt{2}\). Since \(a_1,b_1,a_2,b_2\in \mathbb{Q}\), \((a_1 a_2+ 2b_1 b_2),(a_2 b_1 + a_1 b_2)\in\mathbb{Q}\). Thus, \(x+y\in K\).\\
\(-x=-a+-b\sqrt{2}\). Since \(a,b\in\mathbb{Q}\), \(-a,-b\in\mathbb{Q}\). Thus, \(-x\in K\).\\
If \(a+b\sqrt{2}\not = 0\), \(a,b\not = 0\), and \(a-b\sqrt{2}\not = 0\). Thus, \(x^{-1}=\frac{1}{a+b\sqrt{2}}=\frac{a-b\sqrt{2}}{a^2-2b^{2}}=\frac{a}{a^2-2b^2}-\frac{b}{a^2-2b^2}\sqrt{2}\). It is easy to see that \textbf{new} \(a,b\in\mathbb{Q}\) as \(a,b\in\mathbb{Q}\). Thus, \(x^{-1}\in K\).
Specifically, if \(a=b=0\), \(0\in\mathbb{Q}\). If \(a=1,b=0\), \(1\in\mathbb{Q}\).\\
Thus, \(K\) is a field.
\item \textbf{Exercise 13}
\label{sec:org6ee357e}
Same technique as Exercise 12.
\item \textbf{Exercise 14}
\label{sec:org8f894cc}
Same technique as Exercise 12.
\end{enumerate}
\subsection{\(\S\) 2}
\label{sec:orgb0ec50d}
\subsubsection{Exercises}
\label{sec:org6c79971}
\begin{enumerate}
\item \textbf{Exercise 1}
\label{sec:orgac3729f}
Using result from \hyperref[org6baf600]{\textbf{Exercise 4}}, easy to prove.
\item \textbf{Exercise 2}
\label{sec:orgaba91eb}
\begin{enumerate}
\item \((1,-1)\)
\label{sec:orgaa508f1}
\item \((\frac{1}{2},\frac{3}{2})\)
\label{sec:org71d4444}
\item \((1,1)\)
\label{sec:org6ec0852}
\item \((3,2)\)
\label{sec:orgd708610}
\end{enumerate}
\item \textbf{Exercise 3}
\label{sec:org569269e}
Same technique as in \textbf{Exercise 2}.
\item \textbf{Exercise 4}
\label{sec:org09a9fa7}
\label{org6baf600}

Following set of equations is an equivalent of \(x(a,b)+y(c,d)=O\),
$$\begin{aligned}
ax+cy&=0 && (1)\\
bx+dy&=0 && (2)\\
\end{aligned}$$
$$\begin{aligned}
(1)\times d-(2)\times c\Rightarrow (ad-cb)x+cdy-cdy &= 0\\
(ad-cb)x&=0\\
\end{aligned}$$
For \(ad-cb\not =0\) part, clearly we shall see that \(x=0\) as \((ad-cb)x=0\). Plugging \(x\) back to \((1)\), we get \(y=0\). Thus, two vectors are linear independent.\\
For \(ad-cb=0\) part, we need to prove that \(x(a,b)+y(c,d)=O\) has solution other than \(x=y=0\).\\
First, suppose \(a,b,c,d\not = 0\). Since \(ad-cb=0\), \(x\in \mathbb{R}\). By applying technique, we could also show \(y\in \mathbb{R}\). Thus, \((a,b),\ (c,d)\) are linear independent.\\
If \(a,b,c,d\not = 0\) does \textbf{NOT} hold. Without lose of generality (for all the possibilities, \(a,d\) and \(c.b\) are interchangeable), consider following scenarios in a \(xy\) -plane,
\begin{enumerate}
\item \(a=0,c=0\)
\label{sec:org573921d}

If \(a=c=0\), \(x,y\in \mathbb{R}\) in \((1)\). Because the \((2)\) is a line in the plane, there must exist some \(x,y\not = 0\).
\item \(a=0,b=0,c=0\)
\label{sec:org839ca32}

Same argument as above, despite the line represented by \((2)\) is a little bit peculiar (it is \(y=0\)).
\item \(a=0,d=0,c=0\)
\label{sec:orgbf7317f}

Same argument as the first, despite the line represented by \((2)\) is a little bit peculiar (it is \(x=0\)).
\item \(a=0,d=0,b=0,c=0\)
\label{sec:org51d1b8c}

Both \((1), (2)\) represent the whole plane, thus, \(x,y\in \mathbb{R}\).
\end{enumerate}
\item \textbf{Exercise 9}
\label{sec:org1bea9ae}
$$\begin{aligned}
\sum\limits_{i=1}^{r} [a_i\cdot (A_i\cdot \sum\limits_{j=i+1}^{r}A_{j})]&=O && \text{All vectors are mutually perpendicular}\\
&=\sum\limits_{i=1}^{r} [(a_i\cdot A_i)\cdot \sum\limits_{j=i+1}^{r}A_{j}]\\
\end{aligned}$$
Since \(\forall A\in \{A_i\}, A\not = O\), it is only possible that every \(a\) is \(0\). Thus, \({A_i}\) are linearly independent.
\end{enumerate}
\end{document}
